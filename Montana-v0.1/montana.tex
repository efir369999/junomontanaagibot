\documentclass[11pt]{article}
\usepackage[utf8]{inputenc}
\usepackage[T1]{fontenc}
\usepackage{times}
\usepackage{geometry}
\usepackage{amsmath}
\usepackage{amssymb}

\geometry{letterpaper, margin=1in}

\setlength{\parindent}{0pt}
\setlength{\parskip}{10pt}

\begin{document}

\begin{center}
{\Large\bfseries Montana: Finality from UTC Boundaries}

\vspace{12pt}

Alejandro Montana\\
github.com/afgrouptime\\
x.com/tojesatoshi

\end{center}

\vspace{12pt}

\textbf{Abstract.}  A purely self-sovereign finality mechanism would allow distributed systems to achieve irreversible consensus without economic security or honest majority assumptions.  Verifiable Delay Functions provide part of the solution, but the main benefits are lost if finality still depends on accumulated computation depth, which favors specialized hardware.  We propose a solution to the finality problem using UTC time boundaries.  The system achieves deterministic finality at fixed intervals by treating UTC boundaries as consensus points.  VDF computation proves participation within a time window, not computation speed.  As long as nodes maintain reasonable clock accuracy, finality occurs every minute regardless of hardware capabilities.  The system requires minimal structure and no external dependencies beyond physics itself.

\section{Introduction}

Distributed systems have come to rely on external mechanisms to achieve finality.  Proof-of-work systems provide probabilistic finality through computational races.  Proof-of-stake systems provide economic finality through slashing conditions.  Byzantine fault tolerant systems provide deterministic finality through supermajority votes.

The problem of course is that all these mechanisms require trust in collective behavior.  Miners must be trusted not to reorg.  Validators must be trusted not to collude.  Supermajorities must be trusted to remain honest.  The cost of violating finality is economic or probabilistic, not physical.

What is needed is a finality mechanism based on physical constraints instead of trust, allowing any node to verify finality independently without relying on external parties.  Time boundaries that are physically impossible to advance would protect against manipulation, and the cost of attack would be time itself.

In this paper, we propose a solution using UTC time boundaries as consensus points.  Each minute boundary serves as a finality checkpoint.  Nodes prove participation through VDF computation, but hardware speed provides no advantage---fast hardware simply waits for the boundary like everyone else.

\section{UTC Finality Model}

Montana uses UTC time boundaries for deterministic finality.  No external time sources required by the protocol---nodes use system UTC with $\pm$1 second tolerance.

Finality occurs every 1 minute at UTC boundaries: 00:00, 00:01, 00:02, etc.

\begin{center}
\begin{tabular}{|l|l|}
\hline
\textbf{Property} & \textbf{Value} \\
\hline
Time source & System UTC \\
Tolerance & $\pm$1 second \\
Finality interval & 1 minute \\
Protocol sync & None required \\
\hline
\end{tabular}
\end{center}

Nodes accept blocks and heartbeats within $\pm$1 second of their local UTC.  This tolerance accommodates minor clock drift without requiring external synchronization.

\section{VDF as Proof of Participation}

VDF proves participation in a time window---not computation speed:

\begin{verbatim}
Node A (fast hardware):  VDF ready at 00:00:25 -> waits -> F1
Node B (slow hardware):  VDF ready at 00:00:55 -> F1
Node C (too slow):       VDF ready at 00:01:02 -> misses F1 -> F2
\end{verbatim}

Hardware advantage eliminated.  Fast hardware waits for UTC boundary like everyone else.

The VDF construction uses iterated hashing:

$$\text{VDF}(\text{input}, T) = \text{SHAKE256}^T(\text{input})$$

where $T = 2^{24}$ iterations.  Verification uses STARK proofs for $O(\log T)$ validation without recomputing $T$ iterations.

\section{Finality Levels}

\begin{center}
\begin{tabular}{|l|l|l|}
\hline
\textbf{Level} & \textbf{Time} & \textbf{UTC Boundaries} \\
\hline
Soft & 1 minute & 1 boundary passed \\
Medium & 2 minutes & 2 boundaries passed \\
Hard & 3 minutes & 3 boundaries passed \\
\hline
\end{tabular}
\end{center}

Each finality checkpoint contains:
\begin{itemize}
\item UTC timestamp (boundary time)
\item Merkle root of all blocks in window
\item VDF proofs from participating nodes
\item Aggregate signatures (SPHINCS+)
\item Previous checkpoint hash
\end{itemize}

\section{ASIC Resistance}

\begin{center}
\begin{tabular}{|l|l|l|}
\hline
\textbf{Scenario} & \textbf{VDF Depth Model} & \textbf{UTC Model} \\
\hline
ASIC vs CPU & 40x advantage & No advantage \\
Finality time & Variable & Fixed (1 min) \\
Attack vector & Faster VDF = more depth & None \\
\hline
\end{tabular}
\end{center}

No hardware advantage can advance UTC.  Time passes equally for all.

\section{Temporal Time Unit}

Montana defines a Temporal Time Unit (TTU):

$$\lim_{\text{evidence} \to \infty} 1\text{ TTU} \to 1\text{ second}$$

Total supply: 1,260,000,000 TTU = 21,000,000 minutes.

Pre-allocation: 0.

\section{Participation}

Two node types:
\begin{itemize}
\item \textbf{Full Node}: Full blockchain history, VDF computation (Tier 1)
\item \textbf{Light Node}: History from connection time only (Tier 2)
\end{itemize}

Three participation tiers with lottery weights:
\begin{itemize}
\item Tier 1 (Full Node): 70\%
\item Tier 2 (Light Node / TG Bot owners): 20\%
\item Tier 3 (TG Community): 10\%
\end{itemize}

\section{Cryptography}

Post-quantum from genesis:

\begin{center}
\begin{tabular}{|l|l|l|}
\hline
\textbf{Function} & \textbf{Primitive} & \textbf{Standard} \\
\hline
Signatures & SPHINCS+-SHAKE-128f & NIST FIPS 205 \\
Key Exchange & ML-KEM-768 & NIST FIPS 203 \\
Hashing & SHA3-256, SHAKE256 & NIST FIPS 202 \\
\hline
\end{tabular}
\end{center}

\section{Privacy}

The necessity to announce all transactions publicly precludes traditional privacy, but privacy can be maintained by keeping public keys pseudonymous.

Optional privacy tiers:

\begin{center}
\begin{tabular}{|l|l|l|}
\hline
\textbf{Tier} & \textbf{Visibility} & \textbf{Fee Multiplier} \\
\hline
T0 & Transparent & 1x \\
T1 & Hidden recipient & 2x \\
T2 & Hidden amount & 5x \\
T3 & Full privacy & 10x \\
\hline
\end{tabular}
\end{center}

\section{Conclusion}

We have proposed a finality mechanism based on UTC time boundaries rather than accumulated computation.  The system achieves deterministic finality at fixed intervals, eliminating hardware advantages that plague other approaches.  VDF computation proves participation within a time window, not speed.  Fast hardware provides no benefit---it simply waits for the boundary.

The key insight is that no adversary can advance UTC.  Time passes equally for all participants regardless of computational resources.  This transforms finality from an economic or computational race into a physical certainty.

Montana requires minimal assumptions: that nodes maintain reasonable clock accuracy (within $\pm$1 second), and that the hash function admits no iteration shortcut.  Both are empirically verified properties of existing systems.

\section*{References}

\begin{enumerate}
\item D. Boneh, J. Bonneau, B. B\"unz, B. Fisch, "Verifiable Delay Functions," 2018.
\item NIST FIPS 202, "SHA-3 Standard," 2015.
\item NIST FIPS 203, "Module-Lattice-Based Key-Encapsulation Mechanism," 2024.
\item NIST FIPS 205, "Stateless Hash-Based Digital Signature Standard," 2024.
\item L. Lamport, "Time, Clocks, and the Ordering of Events," 1978.
\item Y. Sompolinsky, A. Zohar, "PHANTOM," 2018.
\item S. Haber, W.S. Stornetta, "How to Time-Stamp a Digital Document," 1991.
\end{enumerate}

\end{document}
